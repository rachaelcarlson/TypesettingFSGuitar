\documentclass[]{memoir}

%%
% Book metadata
\title{Typesetting Finger-Style Guitar}
\author{Rachael Carlson}

%%
% For nicely typeset tabular material
\usepackage{booktabs}

%%
% Prints a trailing space in a smart way
\usepackage{xspace}

%%
% Font configurations
\usepackage{fontspec}
\defaultfontfeatures{Mapping=tex-text}
\usepackage{xunicode}
\usepackage{xltxtra}

\setmainfont[Numbers=OldStyle]{Minion Pro}
\setsansfont[Numbers=OldStyle,
BoldFont=Myriad Pro Black]{Myriad Pro}
\setmonofont[Scale=MatchLowercase]{Anonymous Pro}

%%
% EToolBox is needed for using monospaced numbers within the tabular
% environment
\usepackage{etoolbox}

%%
% Use monospaced numbers within the tabular environment
\AtBeginEnvironment{tabular}{\setmainfont[Numbers=Monospaced]{Minion Pro}}

\usepackage{hyperref}
% \hypersetup{
%     colorlinks,
%     citecolor=black,
%     filecolor=black,
%     linkcolor=red,
%     urlcolor=black
%   }

\usepackage{graphicx}

%%
% Generates the index
\usepackage{makeidx}
\makeindex

\chapterstyle{madsen}

%%
% Style parts like chapterstyle
% Thanks Gonzalo Medina from TeX.stackexchange.com
\renewcommand\partnamefont{\normalfont\Large\scshape}
\renewcommand\partnumfont{\normalfont\Large\scshape}
\renewcommand\parttitlefont{\normalfont\Huge\bfseries\sffamily}
\setsecheadstyle{\normalfont\Large\bfseries\sffamily}

%%
% color codes This doesn't work
% \usepackage{minted}
% \usemintedstyle{tango}

\begin{document}
\frontmatter

% Title Page
\pagestyle{empty}
\begin{flushright}
  Rachael Carlson\\
\end{flushright}
\vspace*{1in}
{\Huge \textsf{\textbf{Typesetting}}}\\

\vspace*{.15in}
\noindent{\Huge \textsf{\textbf{Finger-Style}}}\\

\vspace*{.15in}
\noindent{\Huge \textsf{\textbf{Guitar}}}\\
\begin{flushright}
  {\Large An Exploration and} \\
\end{flushright}
\begin{flushright}
  {\Large Attempt at Codification of} \\
\end{flushright}
\begin{flushright}
  {\Large Typesetting Techniques}\\
\end{flushright}
\begin{flushright}
  {\Large For Finger-Style Guitar}\\
\end{flushright}

\vspace*{1in}

\url{rachaelcarlson.com}

\clearpage
\newpage
  ~\vfill
  \thispagestyle{empty}
  % \setlength{\parindent}{0pt}
  % \setlength{\parskip}{\baselineskip}
  \noindent Copyright © 2018 Rachael Carlson\\

  \noindent \textsc{Published by Rachael Carlson}\\

  \noindent \textsc{https://www.rachaelcarlson.com}\\

  \noindent Licensed under the Apache License, Version 2.0 (the
  ``License''); you may not use this file except in compliance with
  the License. You may obtain a copy of the License at \url{http://www.apache.org/licenses/LICENSE-2.0}.
  Unless required by
  applicable law or agreed to in writing, software distributed under
  the License is distributed on an \textsc{``as is'' basis, without
    warranties or conditions of any kind}, either express or
  implied. See the License for the specific language governing
  permissions and limitations under the License.\index{license}

  \par\textit{Alpha Edition, \today}

  \clearpage

  %% Contents
\tableofcontents


%% List of figures (we will eventually have some figures to add
% \listoffigures

\listoftables

\clearpage
~\vfill
{\Huge \textit{For John Stropes, Joshua Lane, and}}\\

\vspace*{.15in}
{\Huge \textit{Meghan Carlson, without whom I wouldn't.}}
\vfill
\vfill

\clearpage
\pagestyle{ruled}
\nouppercaseheads

\chapter{Introduction}
This document owes much to the work of John Stropes and Joshua Lane at
the University of Wisconsin-Milwaukee. Depending on one's view, this
document could be considered lovingly stolen from the stylesheet used
by John Stropes at Stropes Editions, Ltd. This work was inspired by
the class \textsc{transcription and typesetting for finger-style
  guitar} at the University of Wisconsin-Milwaukee in the Spring of
2017.

The purpose of this document is to generate consistency in the
production of finger-style guitar scores. I offer it freely to all who
are interested. If you feel motivated to contribute in any way, please
feel free to email me your questions, comments, and concerns. This
offering is done solely based upon my desire to see the transcription
and typesetting of finger-style guitar music rise above the mundane
and become a thing of beauty, clarity, and conviction. I believe that
this can be achieved through special attention to detail. For
instance, a quarter of a point can make the difference between the
annotation of a `\textsf{p}' clarifying or obfuscating the will of the
typesetter.

\mainmatter

\part{Finale}
\chapter{Starting a New Finale Document}
\label{cha:start-new-docum}

Before beginning a new Finale document one must ensure that all
elements of the composition are known. These elements include but are
not limited to its form, key signature, time signatures, desired staff
for standard notation, etc. Knowing these elements will ensure that
the preparation of the new document will keep bugs away later
on. Changing these elements after having started the document can
introduce bugs which will either greatly hinder the development of the
document or make it impossible to continue.

\section{Initial Steps}
\label{sec:initial-steps}

The first several steps for creating a new document can seem
tedious. At this point it seems that there is a possibility to create
a template for each desired document layout. After creating a new
document from a template one would then load a Finale Library in order
to ensure consistency across multiple documents.

Begin by starting a \emph{New Document}. This \emph{New Document} will
be a \emph{Default Document} as opposed to a document created through
the \emph{Setup Wizard}. This new document will be have around 32
measures of a treble clef with a time signature of 4/4 and a key
signature of C major. In the Windows 10 version of Finale these will
need to be deleted in order for the page margins to properly update.

Next, change the margins. Changing the margins of the document at this
early stage will help to ensure that the number of measures per system
will be appropriate. Changing the number of measures per system after
annotations such as left-hand duration have been started can result in
hours of more work to correct such a small oversight.

In Finale for MacOS select the \emph{Page Layout Tool}. From the menu
bar, open the \emph{Page Layout Tool}, go to \emph{Page Margins} and
\emph{Edit Page Margins}. For the top and bottom of the page, we will
use 36pt. For the left and right of the page, we will use 54pt. Ensure
that All Pages is selected within the Change parameter. Select Apply
to Parts/Score and select Score and Parts.

In Windows 10 there is a bug which seems to make it impossible to
change the page margins in the same manner explained
above.\footnote{Bug noted as of April, 2018.} The workaround is to go
to \emph{Document} in the menu. Within this menu go to \emph{Page
  Format} then \emph{score}. Under the Page Margins section select
36pt for Top and Bottom and 54pt for Left and Right. Perhaps a related
bug requires several more steps in order to update the page layout. Go
to Page Layout. Click on the \emph{Update Page Layout Tool}. Then
delete the existing default measures. This should update the page
layout.

\section{Score Manager}
\label{sec:score-manager}

\textbf{Window $\rightarrow$ Score Manager}

\begin{itemize}
\item Instrument List
  \begin{enumerate}
  \item Standard Notation
    \begin{enumerate}
    \item Change clef for blank staff to desired clef
    \item Add additional standard notation clefs as needed and change
      clef
    \end{enumerate}
  \item Tablature
    \begin{enumerate}
    \item Add Instrument $\rightarrow$ Tablature $\rightarrow$ Guitar [TAB with Stems]
    \item Change clef to serif `TAB'
    \item Notation Style $\rightarrow$ Tablature $\rightarrow$ Settings
      \begin{enumerate}
      \item Change the tuning
        \begin{enumerate}
        \item Edit Instruments
        \item Change pitches of each string to document tuning
          \begin{itemize}
          \item These will be \textsc{midi} note pitches
          \item C\textsubscript{4} = 60
          \end{itemize}
        \end{enumerate}
      \item Default Lowest Fret = 0
      \item Capo Position = 0
      \item Options
        \begin{enumerate}
        \item Show Tuplets \textbf{checked}
        \item Show Clef Only On First Measure \textbf{unchecked}
        \end{enumerate}
      \item Fret Numbers
        \begin{enumerate}
        \item Vertical Offset = -4
        \item Appearance
          \begin{itemize}
          \item Use Letters \textbf{unchecked}
          \item Break Tablature Lines at Numbers \textbf{checked}
          \end{itemize}
        \end{enumerate}
      \end{enumerate}
    \end{enumerate}
  \end{enumerate}
\item File Info
  \begin{enumerate}
  \item Add all pertinent information in each of the respective sections 
  \end{enumerate}
\end{itemize}

\section{Measures}
\label{sec:measures}

\begin{enumerate}
\item Edit $\rightarrow$ Add Measures
\item Add the total number of measures for the document
  \begin{itemize}
  \item If the document has multiple movements add the total number for all movements
  \end{itemize}
\end{enumerate}

\section{Time and Key Signatures}
\label{sec:time-key-signatures}

\begin{enumerate}
\item Change the time signature for the document
  \begin{itemize}
  \item If the document has multiple changes in time signature, add all of these changes in the appropriate measures
  \end{itemize}
\end{enumerate}

\chapter{Document Settings}
These are the settings which are configurable through the
document settings dialog.
\section{T A B Clef}
Edit default TAB clef in Clef Designer (nudge `A' and 	`B' down)\\
Font: TeXGyreSchola, Bold, 10pt
\section{Line Weights}
\begin{table}
  \begin{center}
\begin{tabular}[h!]{l l}
  Line & Weight\\\hline
  Barlines & 5 EVPU\\
  Ledger Lines & 4.6 EVPU\\
  Left Half Ledger Line Length & 7 EVPU\\
  Right Half Ledger Line Length & 7 EVPU\\
  Stems & 2.2 EVPU\\
  Crescendos & 4.2 EVPU\\
\end{tabular}
\end{center}
\caption{Line Weights in EVPU}
\end{table}
\section{Ties}
\paragraph{Placement: Over/Inner}
\begin{tabular}{l l l}
Horizontal: & Start: $-1$pt & End: 1\\
Vertical: & Start: 3 & End: 3\\
\end{tabular}
\paragraph{Placement: Under/Inner}
\begin{tabular}{l l l}
Horizontal: & Start: 9.5 & End: $-9.5$\\
Vertical: & Start $-3$ & End: $-3$
\end{tabular}
\section{Tablature Slides}
\subsection{Smart Shape Tool}
Smart Shape Placement\\
Tab Slide\\
Same V, Lines, Pitch Increasing\\
\begin{tabular}{l l l l}
Start Point: & H: 1.5pt & End Point: & H: $-1.5$\\
& V: 4 & & V: $-5.5$\\
\end{tabular}
\section{Page Layout}
\subsection{Margins}
Left, Right: 54pt\\
Top, Bottom: 36pt
\chapter{Text}
\section{Title}
Font: Avenir Next Heavy, 28pt\footnote{A bug in Finale 25 on Windows
  10 makes it so that you have to type in the name of the font exactly
  in order for the font to appear on screen. To embed the font for
  print you have to Print to PDF. To do so, in the print dialog,
  choose the Microsoft Print to PDF printer.}

\paragraph{Frame Attributes}
Inserted preset text box (editable through score manager), page 1 only\\
Horizontal: Center, 0pt\\
Vertical: Top (Header), 0\\
Position From: Page Margin\\
Position from Edge of Frame: \textbf{checked}

\section{Subtitle}
\label{sec:subtitle}

Font: Avenir Next, Regular, 8pt

\paragraph{Frame Attributes}
\label{sec:frame-attributes}

Inserted preset text box (editable through Score Manager), page 1 only\\
Centered, Top, Page Margin; H: 0, V: $-32$pt\\
Position from edge of frame: \textbf{checked}

\section{Tuning}
\label{sec:tuning}

Font: Pitches, TeXGyreSchola, Regular, 10pt\\
\indent Octave designations: TeXGyreSchola, Regular, 6pt\\
\indent Baseline shift: -1\\
Accidentals: TeXGyreSchola, Regular, 8pt\\
\indent Superscript: 2

\paragraph{Frame Attributes}
\label{sec:frame-attributes-1}

Text box, page 1 only\\
Horizontal: Left, 1\\
Vertical: Top (Header), -36\\
Position from edge of frame: \textbf{checked}

\section{Composer}
\label{sec:composer}

Font: TeXGyreSchola, Regular, 10pt\\

\paragraph{Frame Attributes}
\label{sec:frame-attributes-2}

Inserted preset text box (editable through Score Manager), page 1 only\\
Horizontal: Right; -1pt\\
Vertical: Top (Header), -36 (align with tuning); Arranger -49\\
Position from: Page Margin\\
Position from edge of frame: \textbf{checked}\\

\section{Copyright}
\label{sec:copyright}

Font: TGS, Regular, 8pt (This is a modified version of TeXGyreSchola with Old Style numerals)

\paragraph{Frame Attributes}
\label{sec:frame-attributes-3}

Inserted preset text box (editable through Score Manager), page 1 only\\
Horizontal: Centered, 0\\
Vertical: Bottom (Footer), -2.25\\
Position from: Page Margin\\
Position from edge of frame: \textbf{checked}\\
Justification: Center

\section{Page Number}
\label{sec:page-number}

Font: TGS, Regular, 8pt

\paragraph{Frame Attributes}
\label{sec:frame-attributes-4}

Inserted preset text boxes: [Title] [File Date] [Page Number]/[Total Pages]\\
Attach to: All Pages\\
Horizontal: Right, 0\\
Vertical: Bottom (Footer), -2.25\\
Position From: Page Margin\\
Position from edge of frame: \textbf{checked}

\section{Timecodes}
\label{sec:timecodes}

Font: Avenir Next, Regular, 8pt

\paragraph{Frame Attributes}
\label{sec:frame-attributes-5}

Text box, Measure attached (standard notation)\\
H: 0\\
V: 48\\
Position from edge of frame: \textbf{checked}

\chapter{Text Expressions}
\label{sec:text-expressions}

\section{Tempo}
\label{sec:tempo}

Justification: Left\\
Horizontal Alignment: Start of Time Signature, 0\\
Vertical Alignment: Staff Reference Line, 36

\section{Time Signatures}
\label{sec:time-signatures}

Font: Maestro, bold, 44pt\\
Justification: Left\\
Horizontal Alignment: Start of Time Signature, 0\\
Vertical Alignment: Staff Reference Line, -22.75

\section{Movements}
\label{sec:movements}

Font: TeXGyreSchola, Italic, 9pt\\
Edit Measure Number Regions\\
One Standard notation staff: Left, Left; H: 1.5, V: -66\\
Grand staff: V: ~-142\\
Show on: Top Staff, \textbf{checked}; Exclude Other Staves, \textbf{checked}; Bottom Staff, \textbf{unchecked}

\chapter{Special Tools}
\label{sec:special-tools}

\section{Beam Angle}
\label{sec:beam-angle}

Eighth note stems: -12\\
Sixteenth note stems: -12\\
Beamed eight notes: -8

\section{Stem Length}
\label{sec:stem-length}

Quarter note stems: -12pt

\chapter{Resize Tool}
\label{sec:resize-tool}

\section{Resize System}
\label{sec:resize-system}

Standard Notation: 85\%\footnote{Click on staff to ensure that you are adjusting the whole staff and not a note.}\\

\noindent Tablature: 90\%

\chapter{Fingerings}
\label{sec:fingerings}

\section{Left-Hand Fingers (Above Staff)}
\label{sec:left-hand-fingers}

% Updated 04/15/2018
Font: TeXGyreSchola, 8pt, \emph{courtesy: 7pt}\\
Enclosure Shape: Circle\\
Line Thickness: 0.44922\\
Height: 10; \emph{courtesy: 9.75}\\
Width: 10; \emph{courtesy: 9.75}\\
Center H: 0\\
V: -0.25\\
Match Height and Width\\
Fixed enclosure size: \textbf{checked}\\
Justification: Center\\
Horizontal: Stem, 2.75; \emph{courtesy: After Clef/Key/Time/Repeat (2.75)}\\
Vertical: Staff Reference Line;\\
First: 12.75pt; Second: 23.75pt; Third: 34.75; Fourth: 45.75

\section{Left-Hand Duration Lines}
\label{sec:left-hand-duration}

When terminated in the same system as its inception, use the \emph{Bracket Tool}.\\
When terminated in a different system than its inception, use the \emph{Line Tool} and make it horizontal.

\paragraph{Elevated Duration Line}
\label{sec:elev-durat-line}

Line Style: Solid; Horizontal, true\\
Thickness: 0.46094\\
End Point Style for elevating duration line one level:\\
\begin{tabular}{l l}
  Start: & End:\\
  Hook, -6.5pt & Hook, -3pt\\
\end{tabular}

\noindent End Point Style for elevating duration line two levels:\\
\begin{tabular}{l l}
  Start: & End:\\
  Hook, -17pt & Hook, -3pt\\
\end{tabular}

\paragraph{Courtesy Parenthesis}
\label{sec:courtesy-parenthesis}

Font: TeXGyreSchola, Regular, 10pt\\
(   ): Three spaces in between each parenthesis\\
Justification: Center\\
Horizontal Alignment Point: After Clef/Key/Time/Repeat: 2.75pt\\
Vertical Alignment Point: Staff Reference Line:\\
\indent First, 12pt; Second, 23pt; Third, 34pt; Fourth, 45pt

\section{Left-Hand Fingerings (Below Staff)}
\label{sec:left-hand-fingerings}

\paragraph{Fourth String}
\label{sec:fourth-string}

Justification: Center\\
Horizontal Alignment: Stem, 2.75pt\\
Vertical Alignment: Staff Reference Line, -17.5pt (quarter), -16.5pt (eighth)

\paragraph{Fifth String}
\label{sec:fifth-string}

Justification: Center\\
Horizontal Alignment: Stem, 2.75pt\\
Vertical Alignment: Staff Reference Line, -26.5pt (quarter), -25.5pt (eighth)

\paragraph{Sixth String}
\label{sec:sixth-string}

Justification: Center\\
Horizontal Alignment: 2.75pt\\
Vertical Alignment: Below Staff Baseline or Entry, -35 (quarter), -34 (eighth)\\

\paragraph{Additional Offsets}
\label{sec:additional-offsets}

Additional Entry Offset:\\
\indent First, -13.75pt; Second, -24.75pt; Third, -35.75pt; Fourth, -46.75pt

\section{Parentheses}
\label{sec:parentheses}

\emph{Note: this is for surrounding a tablature notehead with a parenthesis. This is used when a finger of the left hand is placed on a fret but the right hand does not play the string.}

Font: Avenir Next, Regular, 10pt\\
(   )-three spaces between for single-digit tablature, four spaces for double-digit tablature\\
Justification: Center\\
Horizontal: Stem, 3pt\\
Vertical: Staff Reference Line\\
\indent First, -2.5pt; Second, -11.5pt; Third, -20.5pt; Fourth, -29.5pt; Fifth, -38.5pt; Sixth, -47.5pt

\section{Right-Hand Fingerings}
\label{sec:right-hand-fing}

Font: TeXGyreSchola, Regular, 8pt\\

\paragraph{Positioning: I, M, A}
\label{sec:positioning:-i-m}

Justification: Center\\
Horizontal Alignment: Stem, -5; two-digit numbers -7\\

\begin{table}[h!]
  \centering
  \begin{tabular}{l l r}
    String & Reference & Alignment\\\hline
    First & Staff Reference Line & 2.25\\
    Second & Staff Reference Line & -7.25\\
    Third & Staff Reference Line & -16.5\\
    Fourth & Staff Reference Line & -25.25\\
\end{tabular}
\caption{Vertical Alignment of i, m, a}
\end{table}

\paragraph{Positioning: P}
\label{sec:positioning:-p}

Justification: Center\\
Horizontal Alignment: Stem, -3.75pt\\

\begin{table}[h!]
  \centering
  \begin{tabular}{l l}
    String & Alignment\\\hline
    First & Staff Reference Line, -6.5\\
    Second & Staff Reference Line, -16.5\\
    Third & Staff Reference Line, -24.25; strum, -33\\
    Fourth & Staff Reference Line, -32.75; strum, -41.75\\
    Fifth & Staff Reference Line, -41.75; strum, -50.75\\
    Sixth & Staff Reference Line, -50.5; strum, -55.75\\
  \end{tabular}
  \caption{Vertical Alignment of p}
  \label{tab:p}
\end{table}

\section{Muted Notes}
\label{sec:muted-notes}

Enclosure Shape: Circle\\
Line Thickness: 0.08984\\
Height: 8.5\\
Width: 8.5\\
Center H: 0\\
V: 0.25\\
Match Height and Width\\
Fixed enclosure size: \textbf{checked}

\chapter{Staff Attributes}
\label{sec:staff-attributes}

Notehead font: Avenir Next Medium, 12pt\\

\section{Stems}
\label{sec:stems}

Always down\\
Horizontal Stem Offsets: 0, 0\\
Use vertical offset for notehead end of stems: \textbf{checked}\\
Offset from noteheads: Up, 6.25pt, Down, -6.25pt\\
Use Vertical offset for beam end of stems (offset from staff) \textbf{unchecked}\\


\chapter{Harmonics}
\label{sec:harmonics}

Enter the number for the harmonic node\\
Special Tools > Note Shape Tool\\

\section{Notehead Settings}
\label{sec:notehead-settings}

\paragraph{Positioning}
\label{sec:positioning}

Horizontal: 0\\
Vertical: 1\\
Allow vertical positioning: \textbf{checked}

\paragraph{Font}
\label{sec:font}

Use default notehead font \textbf{unchecked}\\
Zeal 9 plain

\paragraph{Surrounding}
\label{sec:surrounding}

`<' and `>' are separate expressions\\
Font: Zeal 9 plain\\

<:\\
\indent Justification: Center\\
\indent Horizontal Alignment Point: Stem\\
\indent Additional Horizontal Offset: -2.75\\

>:\\
\indent Justification: Center\\
\indent Horizontal Alignment Point: Stem\\
\indent Additional Horizontal Offset: 9\\

\begin{table}[h!]
  \centering
  \begin{tabular}{l l}
    String & Alignment\\\hline
    First & -3pt\\
    Second & -12\\
    Third & -21\\
    Fourth & -30\\
    Fifth & -39\\
    Sixth & -48\\
  \end{tabular}
  \caption{Vertical Alignment of < and >}
\end{table}

\part{LilyPond}
\label{part:lilypond}

\chapter{Brief Introduction to LilyPond}
\label{cha:brief-intr-lilyp}

LilyPond\footnote{\url{lilypond.org}} is a free and open-source music
typesetting program. As it pertains to the typesetting of finger-style
guitar, LilyPond offers a great deal to those with patience. I
personally found my way to LilyPond after a thorough search of the
free and low-cost music typesetting programs. Soon after my
preliminary experiments with LilyPond it became apparent to me that
this program could reproduce, with a great deal of accuracy, the
transcriptions produced by Stropes Editions,
Ltd.\footnote{\url{stropes.com}} LilyPond describes itself as a
\begin{quotation}
  \noindent music engraving program, devoted to producing the
  highest-quality sheet music possible.  It brings the aesthetics of
  traditionally engraved music to computer printouts. LilyPond is free
  software and part of the GNU Project.\footnote{LilyPond... Music
    Notation for Everyone. \url{http://lilypond.org/} Accessed May 25,
    2018.}
\end{quotation}
I find several components of this compelling. First, and truly most
important, is that the program and those dedicated few who contribute
to its development are interested in producing the best quality
engravings. Once one dives deep into this program it is easy to
perceive this devotion. Secondly is that the program is free and
cross-platform. On a more personal level, I enjoy that I am able to
use my favorite text editor,
\href{https://www.gnu.org/software/emacs/}{Emacs}, to create sheet
music.

LilyPond can be a daunting program when first encountered due to its
steep learning curve. I would highly encourage anyone interested to
read LilyPond's
\href{http://lilypond.org/introduction.html}{introduction} with
special attention to the section on
\href{http://lilypond.org/text-input.html}{Text input}. This is
perhaps the most radical component of this software. The engraver
produces a simple text document then runs the LilyPond program on that
file. Depending on what is contained within the file, LilyPond can
produce a variety of document types, such as \textsc{eps},
\textsc{png}, and \textsc{pdf}.

This part of the book will not attempt to introduce the reader to how
to produce scores with LilyPond. Rather, this document will serve as a
guidepost for creating accurate finger-style guitar scores with
LilyPond.

A couple of notes: first, the defaults for guitar tablature in
LilyPond seem to loosely follow the conventions employed by Hal
Leonard. If those conventions are considered good enough for you, I
would recommend diving right in to using LilyPond as a music engraving
tool. Second, LilyPond is attempting to recreate, in digital format,
the sheet music produced by Barenreiter Verlag in the beginning of the
20th century. Due to this LilyPond can seem a bit heavier handed than
a digital engraving program like Finale or Sibelius. It is possible to
use \href{https://www.smufl.org/}{SMuFL} fonts to change this
heaviness. This book will not be going into the use of these fonts.

My hope is that by producing this component of this book I will be
able to encourage more individuals to use LilyPond to create their
finger-style guitar scores. While LilyPond may not have some of the
features of Finale or Sibelius at the moment, its price tag of free
and this introduction into using it for finger-style guitar may keep
users away from cheap alternative music engraving software that are
woefully lacking in the necessary features to create an accurate,
detailed, and beautiful finger-style guitar score.

This section of the book will contain examples of LilyPond code. As
mentioned above, the expectation is the reader will have availed his
or herself of the Introduction to LilyPond. A good editor to start
with for editing LilyPond code would be
Frescobaldi.\footnote{\url{http://frescobaldi.org/uguide.html}}

\chapter{Starting a New LilyPond Document}
\label{cha:start-new-lilyp}

\section{The Paper Environment}
\label{sec:paper-environment}

Within the paper environment there are a number of variables which
need to be noted as the default variables are not going to work for
finger-style guitar. Namely, the paper size, margins, fonts, and
systems-per-page.
\begin{verbatim}
\paper {
  #(set-paper-size "letter" 'portrait) % 'landscape is also an option
  left-margin = 54\pt 
  right-margin = 54\pt
  top-margin = 36\pt
  bottom-margin = 36\pt
  max-systems-per-page = 3 % Any number you want
  min-systems-per-page = 3 % Sometimes LilyPond doesn't put enough 
                           % systems on a page
  myStaffSize = #20 % This is a little smaller than default
  #(define fonts
    (set-global-fonts
      #:music "mtf-haydn"
      #:roman "Old Standard"
      #:sans "TeX Gyre Heros"
    )
  )
}
\end{verbatim}

\backmatter
\printindex

\end{document}

%%% Local Variables:
%%% mode: latex
%%% TeX-master: t
%%% End:
